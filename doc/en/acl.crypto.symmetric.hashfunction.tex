\chapter{Crypto.Symmetric.Hashfunction}\label{Hash}
In this generic package a cryptographic hash function is generated
from the algorithm of a cryptographic hash function (Chapter
\ref{AlgorithmHash}), which is used for the following purposes:
\begin{itemize}
\item Integrity testing of messages
\item Creation and verification of digital signatures
\item Creation of random numbers or random bits
\end{itemize}
The purpose of this package is to standardize and simplify the API for
hash functions.
\section{API}
\subsubsection*{Generic Part}
\begin{lstlisting}{}
  generic
   type Hash_Type                 is private;
   type Message_Type              is private;
   type Message_Block_Length_Type is range <>;
   with function Generic_To_Bytes(DWord_Array : Hash_Type)
   								      return Bytes is <>;
   with procedure Init(Hash_Value : out Hash_Type) is <>;
   with procedure Round(Message_Block: in     Message_Type;
                        Hash_Value   : in out Hash_Type) is <>;
   with function Final_Round(Last_Message_Block : Message_Type;
                 Last_Message_Length : Message_Block_Length_Type;
                 Hash_Value  : Hash_Type) return Hash_Type is <>;
   with procedure Hash(Message    : in   Bytes;
                       Hash_Value : out  Hash_Type) is <>;
   with procedure Hash(Message    : in   String;
                       Hash_Value : out  Hash_Type) is <>;
   with procedure F_Hash(Filename   : in   String;
                         Hash_Value : out  Hash_Type) is <>;
\end{lstlisting}
The API of a generic hashfunction is made of a High- and a
Low-Level-API. The Low-Level-API should be only used when the user is
familiar with the cryptographic hashfunction. If it's not in this
situation, then please only use the High-Level-API.

\subsubsection*{High-Level-API}
\begin{lstlisting}{}
  function Hash  (Message  : Bytes)  return Hash_Type;
  function Hash  (Message  : String) return Hash_Type;
  function F_Hash(Filename : String) return Hash_Type;
\end{lstlisting}
The function \texttt{Hash()} returns a hash value of a message. The
type of the message can either be bytes or a string. The function
\texttt{F\_Hash()} works on a file. For example:
$H:=F\_Hash("/bin/ls")$; returns a hash value of "$/bin/ls$".

\subsubsection*{Low-Level-API}
The Low-Level-API consists of the following operations.
\begin{itemize}
\item One procedure \texttt{Init()} initializes or reinitializes the
  hash function. Every time before a message is to be hashed, this
  procedure should be called.
\begin{lstlisting}{}
  procedure Init;
\end{lstlisting}
\item One procedure \texttt{Round()} can be called iteratively to hash
  message blocks.
\begin{lstlisting}{}
  procedure Round(Message_Block : in Message_Type);
\end{lstlisting}
\item Function \texttt{Final\_Round()} pads and hashs a message block
  \texttt{Last\_Message\_Block}. Because of padding, the length of the
  exact message content \texttt{Message\_Block\_Length\_Type} should
  be specified in byte. Usually a message is shorter than a message
  block of type \texttt{Message\_Type}. The returning value of the
  function is corresponding to the final hash value of the message.
\begin{lstlisting}{}
 function Final_Round(Last_Message_Block:Message_Type;
        	    Last_Message_Length:Message_Block_Length_Type)
        			 	  			 			return Hash_Type;
\end{lstlisting}
\end{itemize}

%%%%%%%%%%%%%%%%%%%%%%%%%%%%%%%%%%%%%%%%%%%%%%%%%%%%%%%%%%%
%%%%%%%%%%%%%%%%%%%%%%%%%%%%%%%%%%%%%%%%%%%%%%%%%%%%%%%%%%%

\section{Example}
\subsubsection*{High-Level-API}
This example gives the SHA-256 hash value of $/bin/ls$.
\begin{lstlisting}{}
  with Ada.Text_IO; use Ada.Text_IO;
  with Crypto.Types; use Crypto.Types;
  with Crypto.Symmetric.Hashfunction_SHA256;

  procedure Example_Hash_HL is
    package SHA256 renames
    							Crypto.Symmetric.Hashfunction_SHA256;
    Hash : W_Block256 := SHA256.F_Hash("/bin/ls");
  begin
    for I in Hash'Range loop
      Put(To_Hex(Hash(I)));
    end loop;
    Put_Line(" /bin/ls");
  end Example_Hash_HL;
\end{lstlisting}
\subsubsection*{Low-Level-API}
\begin{lstlisting}{}
  with Ada.Text_IO;   use Ada.Text_IO;
  with Crypto.Types;  use Crypto.Types;
  with Crypto.Symmetric.Hashfunction;
  with Crypto.Symmetric.Algorithm.SHA256;
  use Crypto.Symmetric.Algorithm.SHA256;
  pragma Elaborate_All(Crypto.Symmetric.Hashfunction);

  procedure Example_Hash_LL is
    package WIO is new Ada.Text_IO.Modular_IO(Word);
    package SHA256 is new Crypto.Symmetric.Hashfunction
    			      (Hash_Type    => W_Block256,
			    		 Message_Type => W_Block512,
		             Message_Block_Length_Type => 
				             Crypto.Types.Message_Block_Length512,
    			       Generic_To_Bytes => Crypto.Types.To_Bytes);
    Message : String :=("All your base are belong to us! ");
    W : Words := To_Words(To_Bytes(Message));
    M : W_Block512 := (others => 0);
    H : W_Block256;
  begin
    for I in W'Range loop
      M(I) := W(I);
    end loop;
    SHA256.Init;
    H := SHA256.Final_Round(M, Message'Last);
    for I in W_Block256'Range loop
      WIO.Put(H(I), Base=>16);
      New_Line;
    end loop;
    New_Line;
  end Example_Hash_LL;
\end{lstlisting}
\subsection*{Remark:}
Users don't need to generate every time a new hash function. There are
already hash functions defined in the ACL.
\begin{itemize}
\item \texttt{Crypto.Symmetric.Hashfunction\_SHA1}
\item \texttt{Crypto.Symmetric.Hashfunction\_SHA256}
\item \texttt{Crypto.Symmetric.Hashfunction\_SHA512}
\item \texttt{Crypto.Symmetric.Hashfunction\_Whirlpool}
\end{itemize}
